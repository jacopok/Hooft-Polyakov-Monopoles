\begin{abstract}
In this work we show the fundamental steps of the approaches by Dirac and by 't Hooft and Polyakov to the theory of magnetic monopoles.

In the former a duality transformation is applied to classical magnetism, a magnetic charge is manually inserted at the origin and the quantization of charge is deduced; in the latter an \emph{ansatz} for the Euler-Lagrange equations of the Georgi-Glashow model --- a nonabelian gauge theory with a Higgs mechanism --- is studied, and found to have a \emph{topological} charge, with a similar quantization condition to the Dirac monopole but without the need to insert a central magnetic charge.

% deducing charge quantization for electromagnetism and generalizing the approach per non abelian gauge theories under Higgs mechanism.
\end{abstract}