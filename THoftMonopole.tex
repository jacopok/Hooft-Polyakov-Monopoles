\section{’t Hooft-Polyakov Monopoles}
It will be showed that free magnetic monopoles in the physical vacuum are possible as regular solutions of the field equations of the Georgi-Glashow model, following the notorious works of Gerard 't Hoft \cite{Hof:Mon} and Aleksander Polyakov \cite{Pol:Mon} in 1974. 
\subsection{An Intuitive Example}
A simple and very clarifing example, which gives an intuitve idea of how it is possible to have monopoles in quantum fied theory, is given in the introduction of \cite{Hof:Mon}. \\
We should consider a two-dimensional spherical surface $S_2$ in the usual 3D space, with some magnetic flux $\Phi$ enetering in a spot. If the spot is surrounded by a contour $C_0$, the potential around $C_0$ suld be equal to: 
\begin{equation}
    \Vec{A}= \nabla \Lambda 
\end{equation}
where $\Lambda$ is a gauge transformation, which acts on the wavefunction in this way: $\psi \rightarrow e^{i\Lambda} \psi $. 
While $\Lambda$ is a multivalued function, $\Phi$, being a physical field, is required to be a single-valued function. Since, 
\begin{equation}
    \Phi = \int_{C_0} \nabla \Lambda \cdot d\Vec{l}
\end{equation}, 
$\Psi$ should be equal to an integer multiplied for $2 \pi  $, a complete gauge rotation along the contour. 
In an Abelian Gauge theory, there must be another spot from which the flux comes out, because the $2k \pi$ rotation cannot be changed into a constant. \\ 
Instead, in a non Abelian gauge theory, with a compact covering group, such as O(3) , a rotation over $4\pi$ can be shifted into a constant without singularity.
It yelds that if we have a non Abelian gauge theory with a compact covering group and such compact covering group has the the electromagnetic group  U(1) as a subgroup, the existance of the magnetic monopole would be allowed without supposing the existance of any singularity anywhere on the sphere.
A theory that satisifies those requirements is the Georgi-Glashow model for the electroweak interaction.

\subsection{The bosonic part of the Georgi-Glashow Model}
The Georgi-Glashow model was proposed as a theory for the electroweak interaction and, in its bosonic part, is based on the gauge group SU(2) which is isomorphic to SO(3).
The lagrangian density $\mathcal{L}$, whose associated action is invariant under gauge transformation belonging to SO(3), contains the Higgs field $\phi = ( \phi^1, \phi^2 , \phi^3)$, which has values in SO(3) and three gauge potentials $\Vec{W}^\mu$,which take values in the Lie algebra of SO(3). 
$\mathcal{L}$ has the following form : 
\begin{equation}
\mathcal{L}= \frac{1}{2}D_{\mu}\phi^a D^\mu \phi^a  -\frac{1}{4} G_{\mu \nu}^a G^{a \ \mu \nu} - \underbrace{\frac{1}{4}\lambda (\phi^2 -a^2)}_{V(\phi)},
\end{equation}
where $\Vec{G_{\mu \nu}}$ is the gauge field-strength and is defined as follows:
\begin{equation}
\Vec{G_{\mu \nu}} = \partial_{\mu}  \Vec{W}^\nu -\partial_{\nu}  \Vec{W}^\mu - e \Vec{W}^\nu \times \Vec{W}^\mu,
\end{equation}
and the operator $D_{\mu}$ is the gauge covariant derivative: 
\begin{equation}
D_\mu \Vec{\phi} = \partial_\mu \Vec{\phi} - e \Vec{W}_\mu \times \Vec{\phi}.
\end{equation}
Let us look at the physical meaning of each term in the lagrangian density:
\begin{itemize}
    \item The term $\frac{1}{2}D_{\mu}\phi^a D^\mu \phi^a $ is the free field term;
    \item The term  $-\frac{1}{4} G_{\mu \nu}^a G^{a \ \mu \nu}$ descirbes the dynamics of the potential $\Vec{W}_{\mu}$;
    \item The term  $V(\phi) $ describes the self interaction of the Higgs field and contains a non negative constant $\lambda$.
\end{itemize}

Now we are interested at the spectrum of the model, which is composed of four bosons,as shown in the following table:

The boson masses can be obtained through perturbation theory, considering an Higgs potential with little fluctuation around the minimum;
\begin{equation}
\vec{\phi} = \vec{a}+ \vec{\chi}. 
\end{equation}
For the sake of simpicity, we chose as a minimum, the vector $\vec{a}= (0,0,a)$. Then, working in the so-called unitary gauge, the fluctuation vector $\vec{\chi}$ can be written in such a way that $\vec{\chi} = (0,0,\chi) $.
At this point, we should expand the Higgs field around $\vec{a}$ in order to read the masses of the bosons from the quadratic term coefficients:
\begin{gather}
V(\phi)\simeq  \frac{1}{2}  \left(\sqrt{2 \lambda} a\right)^2 \vec{\chi}^2 = \left( \frac{M_\chi}{\hbar }\right)^2 \vec{\chi}^2 \\
 \frac{1}{2}D_{\mu}\vec{\phi} \cdot  D^\mu \vec{\phi} \simeq  \frac{1}{2}D_{\mu}\vec{a} \cdot  D^\mu \vec{a}  + \frac{1}{2} \left( ea \right)^2 =\frac{1}{2}D_{\mu}\vec{a} \cdot  D^\mu \vec{a}  + \frac{1}{2} \left( \frac{M_W}{\hbar} \right)^2
\end{gather}

To deduce the charges of the bosons, instead, we should see how the covariant derivative, which describese the minimal coupling for the photon $A_{\mu}$: 
\begin{equation}
\nabla_\mu = \partial_\mu + i \frac{Q}{\hbar} A_{\mu}
\end{equation}
is embedded in the SO(3) covariant derivative. Then, the value given to $Q$ for each boson, represents the charge of the bosons. 

Another aspect of the Georgi-Glashow model, which is relevant for understanding the 't Hooft-Polyakov monopoles, is the study of the vacuum configurations.\\
Using Noether theorem, the stress energy tensor can be calculated. To be more specific, we are interested at its component $T^{00}$, which describes the energy density: 
\begin{equation}
T^{00} = \frac{1}{2} \Vec{E}^i \cdot  \Vec{E}^i + \frac{1}{2} \Vec{B}^i \cdot \Vec{B}^i +\frac{1}{2} D_0 \Vec{\phi} \cdot D_0 \Vec{\phi} +\frac{1}{2} D_i \Vec{\phi} \cdot  D_i \Vec{\phi} + V(\phi),
\label{eq:Den}
\end{equation}
where $\Vec{E}^i = - \Vec{G}^{0i}$ and $D_0 \Vec{\phi}$ are the conjugate momenta to the gauge field $\Vec{W}_{\mu}$ and $\Vec{B}^i$ is defined through this relation: $ \Vec{G}_{ij} = \epsilon_{ijk} \Vec{B}_k$.\\
Imposing that the energy density $T_{00}$ contained in expression \ref{eq:Den}, is null, we get the conditions for the vacuum configurations: 
\begin{equation}
\Vec{G}_{\mu \nu} = 0 \qquad D^{\mu} \Vec{\phi}  = 0 \qquad V(\phi)=0
\end{equation}
The last two of those conditions give the so-called \textit{Higgs Vacuum} configurations. 
In the Higgs vacuum, the Higgs field is such that $\phi^2 = a^2$. It follows that such vacuum configurations are not invaiant under the whole SO(3) but under the subgroup SO(2)$\simeq$ U(1), which is the gauge group of electromagnetism. Hence, the Georgi-Glashow model undergoes a spontaneous symmetry breaking process.





\subsection{The 't Hooft-Polyakov Ansatz}
\subsection{Topological Origin of Magnetic Charge}
\subsection{Relation with Dirac Monopole}