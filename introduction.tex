\documentclass[main.tex]{subfiles}
\begin{document}

\section{Introduction}

%Electric charge quantization has been proved experimentally by R.A. Millikan in 1909: from that date, all the experiments that followed confirmed it.
Even though they have never been experimentally observed, magnetic monopoles have been a subject of study ever since they were first hypothesized by Pierre Curie in 1894 \cite{Curie:Monopole}.
The theoretical importance of such an idea became evident after the work of Dirac in 1931 \cite{Dirac}: he showed that, postulating the existence of the magnetic monopole and assuming that Maxwell's equations are invariant under a duality transformation, it was necessary to assume the quantization of electric and magnetic charge.
The quantization of electric charge had been experimentally proved by Millikan in 1909, but couldn't be theoretically explained in the framework of classical electromagnetism.
In section \ref{sect:Dirac}, we will show how the Dirac quantization condition can be deduced and how it can be generalised to the Zwanziger-Schwinger condition for particles carrying both electric and magnetic charge: dyons.

Many years later, in 1975, Gerardus 't Hooft \cite{Hof:Mon} and Aleksandr Polyakov \cite{Pol:Mon}, shed new light on the theory of magnetic monopoles.
They separately showed that monopoles emerge as regular solutions of the field equations of the Georgi-Glashow model, which is an SU(2) gauge theory with a Higgs potential.
Their reasoning can be used in any non Abelian gauge theory with compact covering group.
With this approach, monopoles are intrinsic to the theory and fulfill the same quantization condition proved by Dirac.
In section \ref{sect:Hooft} we will expose the derivation of the 't Hooft Polyakov ansatz for the solutions to the Euler-Lagrange equations of the Georgi-Glashow model.
Then, we will discuss the existence of a charge quantization condition for dyons in the framework of the Georgi-Glashow model, following the work of B.\ Julia and A.\ Zee \cite{Julia:Dyon}.
Lastly, we will derive the Bogomol'nyi (lower) bound for the mass of a dyon, discuss its saturation and apply it to the 't Hooft-Polyakov monopole case. 

% \todo[inline]{Missing part: however, electric charge quantization has no theoretical justification...}

%\texthl{In fact}\todo{Why? this is the first time monopoles are mentioned in this section...}, Maxwell's equations in their classical formulation do not formally admit the existence of magnetic monopoles and to date we have no proof of their empirical existence.
%Nonetheless, several physicists (Pierre Curie for the first time) reflected about the reason of asymmetry between electric and magnetic charge in nature.
%In 1931 Dirac showed that by applying an appropriate duality transformation to Maxwell'a equations one can derive  magnetic and electric charge quantization, \texthl{which implies}\todoleft{The quantization of the charge implies the formal possibility of magnetic monopoles?} the formal possibility for the existence of magnetic monopole in a dual electromagnetic theory.
% A different approach has been introduced in QFT by 't Hooft and \texthl{Polyakov} in 1975, deducing charge quantization in a general non abelian gauge theory.
% This approach is different from Dirac one, since it obtains quantization from just the lagrangian action of the theory considered and some reasonable physical assumptions, in both the static and dynamical case.
% \texthl{Nonetheless}\todoleft{What is the issue we can work in spite of?}, it has been shown that this theory can explain \texthl{Dirac results}\todo{What are these results?} for QED, but can \texthl{go further}\todo{Is this in our work specifically?} and explain quantization, for instance, for electroweak mechanism and other non abelian gauge theory.

% In this paper we will recall the fundamental steps of monopolar theories, starting from Dirac quantization for the electromagnetic case and then...
\end{document}