\documentclass[main.tex]{subfiles}
\begin{document}

\section{Introduction}
Electric charge quantization has been proved experimentally by R.A. Millikan in 1909: from that date, all the experiments that followed confirmed this hypothesis.
In fact, Maxwell's equations in their classical formulation do not formally admit the existence of magnetic monopole and since today we have no proof of its empirical existence.
Nonetheless, several physicists (Pierre Curie for the first time) reflected about the reason of asymmetry between electric and magnetic charge in nature.
In 1931 Dirac showed that from an appropriate duality operation in Maxwell equations it follows magnetic charge quantization, which implies formal possibility for the existence of magnetic monopole in a dual electromagnetic theory.
A different approach has been introduced in QFT by 't Hooft and Poljakov in 1975, deducing charge quantization in a general non abelian gauge theory.
This approach is different from Dirac one, since it obtains quantization from the lagrangian action of the theory considered and some reasonable physical assumptions, in both static and extended dynamic case.
Nonetheless, it has been shown that this theory can explain Dirac results for QED, but can go further and explain quantization, for instance, for electroweak mechanism and other non abelian gauge theory.

In this paper we will recall the fundamental steps of monopular theories, starting from Dirac quantization for the electromagnetic case and then...
\end{document}