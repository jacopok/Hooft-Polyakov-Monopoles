\documentclass[main.tex]{subfiles}
\begin{document}

\section{Introduction}

Electric charge quantization has been proved experimentally by R.A. Millikan in 1909: from that date, all the experiments that followed confirmed it.

\todo[inline]{Missing part: however, electric charge quantization has no theoretical justification...}

\texthl{In fact}\todo{Why? this is the first time monopoles are mentioned in this section...}, Maxwell's equations in their classical formulation do not formally admit the existence of magnetic monopoles and to date we have no proof of their empirical existence.
Nonetheless, several physicists (Pierre Curie for the first time) reflected about the reason of asymmetry between electric and magnetic charge in nature.
In 1931 Dirac showed that by applying an appropriate duality transformation to Maxwell'a equations one can derive  magnetic and electric charge quantization, \texthl{which implies}\todoleft{The quantization of the charge implies the formal possibility of magnetic monopoles?} the formal possibility for the existence of magnetic monopole in a dual electromagnetic theory.
A different approach has been introduced in QFT by 't Hooft and \texthl{Polyakov}\todo{La traslitterazione dal cirillico è arbitraria, ma usiamone una sola...} in 1975, deducing charge quantization in a general non abelian gauge theory.
This approach is different from Dirac one, since it obtains quantization from just the lagrangian action of the theory considered and some reasonable physical assumptions, in both the static and dynamical case.
\texthl{Nonetheless}\todoleft{What is the issue we can work in spite of?}, it has been shown that this theory can explain \texthl{Dirac results}\todo{What are these results?} for QED, but can \texthl{go further}\todo{Is this in our work specifically?} and explain quantization, for instance, for electroweak mechanism and other non abelian gauge theory.

In this paper we will recall the fundamental steps of monopolar theories, starting from Dirac quantization for the electromagnetic case and then...
\end{document}