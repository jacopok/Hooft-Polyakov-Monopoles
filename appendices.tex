\documentclass[main.tex]{subfiles}
\begin{document}

\appendix

\section{Homotopy} \label{sec:homotopy}

If we have a topological space \(X\), the set of all maps \(f\) from \([0,1]^n\) such that  \(f(\vec{0}) = f(\vec{1} )=x_0 \in X\) considered modulo homotopy and with the operation of composition is called the \(n\)-th homotopy group \(\Pi_n (X; x_0)\). The dependence on \(x_0\) is actually trivial, it only matters to which connected component of \(X\) the element \(x_0\) belongs.

Composition is to be considered as such:

\begin{equation}
  (f+g) (\vec{x} ) = \begin{cases}
  f(2x_1, x_2, \dots, x_n) \qquad & x_1 \in [0, 1/2 ] \\
  g(2x_1-1, x_2, \dots, x_n) \qquad & x_1 \in [1/2, 1] \,.
  \end{cases}
\end{equation}

Some notable cases are: \(\Pi_n (S^m) = 0 \) for \(n<m\) and \(\Pi_n (S^n) = (\mathbb Z, +)\).


\section{Differential forms and Poincaré's lemma} \label{sec:differential-forms}



\end{document}
