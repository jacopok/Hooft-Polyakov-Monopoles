\documentclass[main.tex]{subfiles}
\begin{document}

\clearpage
\appendix

\section{Differential forms and Poincaré's lemma} \label{sec:differential-forms}

In this appendix we present a brief overview of differential forms, following \cite{Lechner}.

Differential forms are a useful tool when working with completely antisymmetric tensors:
in a space of dimension \(d\) one can define tensors of any rank \(n\), but if we restrict ourselves to the antisymmetric ones, which satisfy:
%
\begin{equation}
  F_{\mu_1 \mu_2 \dots \mu_{n}} 
 = F_{[\mu_1 \mu_2 \dots \mu_{n}]}
\end{equation}
%
we can see that they only have \(\dbinom{d}{n}\) independent components: they can be nontrivial only if \(n \leq d\).  

The differential form associated with the tensor \(F_{\mu_1 \dots \mu_{n}}\) is defined as:

\begin{equation}
  F = \frac{1}{n!} F_{\mu_1 \dots \mu_{n}} \dd{x^{\mu_1}} \wedge \dots \wedge \dd{x^{\mu_n}}\,,
\end{equation}
%
where \(\dd{x^{\mu_1}} \wedge \dots \wedge \dd{x^{\mu_n}}\) is the canonical basis of the exterior algebra \(\bigwedge^n (V)\), where \(V\) is the vector space over which the tensors are constructed.
This basis is antisymmetric: \(\dd{x^{\mu_1}} \wedge \dots \wedge \dd{x^{\mu_n}} = \dd{x^{[\mu_1}} \wedge \dots \wedge \dd{x^{\mu_n]}}\).

The wedge product is an antisymmetrized tensor product: it maps a \(p\)-form \(A\) and a \(q\)-form \(B\) to a \((p+q)\)-form \(A \wedge B\), defined by:
\begin{equation}
  A \wedge B = \frac{1}{p!q!} 
  A_{[\mu_1 \dots \mu_{p}} B_{\nu_1 \dots \nu_{q}]}
  \dd{x^{\mu_1}} \wedge \dots \wedge \dd{x^{\mu_p}} \wedge
  \dd{x^{\nu_1}} \wedge \dots \wedge \dd{x^{\nu_q}}\,.
\end{equation}

It is \emph{graded} antisymmetric: \(A \wedge B = (-)^{pq} B \wedge A\).

The number of independent components tells us that \(n\)-forms and \(d-n\)-forms have the same number of independent components: it is natural to find a connection between them. This is realized by \emph{Hodge duality}: an \(n\)-form \(F\) is mapped to the \((d-n)\)-form \(*F\), defined by:
\begin{equation}
  *F = 
  \frac{1}{n!(d-n)!} \varepsilon_{\mu_1 \dots \mu d} F^{\mu_1 \dots \mu_n} \dd{x^{\mu_{n+1}}} \wedge \dots \wedge \dd{x^{\mu_{d}}}\,.
\end{equation}

The exterior differential maps an \(n\)-form \(F\) to an \((n+1)\)-form \(\dd{F}\) as such:
consider the derivative operator \(\partial \equiv \dd{x^{\mu}} \partial _\mu\).
Then, \(\dd{F} \equiv \partial \wedge F\).

The functions usually considered are sufficiently regular, therefore by Schwarz's theorem \(\dd^2 = 0\).

An \(n\)-form \(F\) can is called \emph{closed} if \(\dd{F}= 0 \) and \emph{exact} if there exists an \((n-1)\)-form \(A\) such that \(F = \dd{A}\).
By \(\dd^2=0\) we have that exact implies closed, while the inverse is not generally true.

A \emph{contractible} subset of a topological space is one which can be continuously deformed to a point.
\emph{Poincaré's lemma} states that, if a form is defined on a contractible set and it is closed, then it is also exact.

The non-contractibiliy of spaces is caused by topological nontriviality: the quotient of the closed \(n\)-forms modulo the exact \(n\)-forms characterizes this and is called the \(n\)-th de Rham cohomology group. 

\section{Homotopy} \label{sec:homotopy}

Given two topological spaces \(X\) and \(Y\) and two functions \(f, g \colon X \rightarrow Y\), the functions are said to be \emph{homotopic} if there exists a continuous deformation \(H \colon X \times [0,1] \rightarrow Y\) such that for any \(x \in X\) we have \(H(x, 0) = f(x)\) while \(H(x, 1) = g(x)\).

Now, consider a topological space \(X\).
The set of all maps \(f\) from the \(n\)-cube \([0,1]^n \rightarrow X\) such that  \(f(t) = x_0 \in X\) if \(t\) is on the boundary of the \(n\)-cube \([0,1]^{n}\), considered modulo homotopy and with the operation of composition is called the \(n\)-th homotopy group \(\Pi_n (X; x_0)\).
The dependence on \(x_0\) is actually trivial: it only matters to which connected component of \(X\) the element \(x_0\) belongs. If the space is arcwise connected, there is no dependence on \(x_0 \) whatsoever.

The composition of two maps \(f\) and \(g\) is defined as:

\begin{equation}
  (f+g) (\vec{x} ) = \begin{cases}
  f(2x_1, x_2, \dots, x_n) \qquad & x_1 \in [0, 1/2 ] \\
  g(2x_1-1, x_2, \dots, x_n) \qquad & x_1 \in [1/2, 1] \,.
  \end{cases}
\end{equation}

The first homotopy classifies loops: for example, for a torus \(\Pi_{1} (\mathbb{T}) = \mathbb{Z}^2\) since we have two independent non-homotopic  ways of going around it, and we can go around the torus any amount of times along either.

Some notable cases for higher homotopy groups are: \(\Pi_n (S^m) = 0 \) for \(n<m\) and \(\Pi_n (S^n) = (\mathbb Z, +)\).

\end{document}
