\documentclass[main.tex]{subfiles}
\begin{document}

\subsection{The Monopole Mass}

We wish to estimate the mass (or, equivalently, energy) of the 't Hooft-Polyakov monopole: in general, the mass of a dyon \todo{Shouldn't it be a monopole?} is given by the integral
%
\begin{equation} \label{eq:monopole-mass}
  E = \int_{\mathbb{R}^3}
  \frac{1}{2} \qty(E_i^a E_i^a + B_i^a B_i^a + D_\mu \phi^a D_\mu \phi^a) + V(\phi)\,.
\end{equation}
%
where \(G_{0i}^a = E_i^a\) and \(G_{ij}^a = B_i^a\)\todo{Definizioni sbagliate!}.

Now, we will derive a lower bound for this mass. Then, we will explicitly show that this bound can be reached by a specific solution: the BPS monopole.

\subsubsection{The Bogomol'nyi Bound}

First of all we drop the (always positive) terms \(D_0 \phi^a D_0 \phi^a\) and \(V(\phi)\).
Then, we apply the famous ``squaring trick'': we add and subtract some terms depending on a generic angle \(\theta\):

\begin{equation}
\begin{split}
    E \geq \frac{1}{2}  \int_{\mathbb R^3}   E_i^a E_i^a + B_i^a B_i^a + D_i \phi^a D_i \phi^a &+2 E_i^a D_i \phi^a \sin \theta - 2E_i^a D_i \phi^a \sin \theta \\
  &+2 B_i^a D_i \phi^a \cos \theta - 2 B_i^a D_i \phi^a \cos \theta \,,
\end{split}
\end{equation}
%
and then we make them into squares, which we can discard while still maintaining the bound:
%

\begin{equation}
\begin{split}
    E \geq  \frac{1}{2} \int_{\mathbb R^3}
    &\qty(E_i^a E_i^a - 2 E_i^a D_i\phi^a \sin\theta + D_i\phi^a D_i\phi^a \sin^2\theta) + \\
    + &\qty( B_i^a B_i^a - 2 B_i^a D_i\phi^a \cos\theta + D_i\phi^a D_i\phi^a \cos^2\theta) +  \\
    +&2 E_i^a D_i \phi^a \sin \theta+2 B_i^a D_i \phi^a \cos \theta \,.
\end{split}
\end{equation}\todo{usare il comando overbrace per scrivere i quadrati in modo esplicito}

The terms in parentheses are positive, therefore the monopole mass is just bounded by:
%
\begin{equation}
  E \geq \int_{\mathbb R^3}
  E_i^a D_i \phi^a \sin \theta+ B_i^a D_i \phi^a \cos \theta
\end{equation}
%
for \emph{any} \(\theta\).
Now, we use the following facts: by the Bianchi identities \(D_\mu \,^* G^{\mu\nu, a} = 0\) we have \(D_i B^a_i = 0\), therefore \(B_i^a D_i \phi^a = D_i(B_i^a \phi^a)\) \todo{formula incomprensibile, abbiamo provato noi a mettere l'uguale}.
Also, the equations of motion read \(D_\nu G^{\mu\nu, a} = e \varepsilon_{abc} \phi^c D^\mu \phi^b\), and \(D_i G_{0i, a} = D_i E_i^a\).  Therefore:
%
\begin{equation}
  D_i \phi^a E_i^a
  = D_i (\phi ^a E_i^a) - \phi^a D_i E^a_i
  = D_i (\phi ^a E_i^a) - \phi^a e \varepsilon_{abc} \phi^c D^\mu \phi^b
  = D_i (\phi ^a E_i^a)\,.
\end{equation}
\
With these two facts we can frame all of the integrand as a divergence:
%
\begin{equation}
  E \geq \int_{\mathbb R^3}
  D_i \qty(E_i^a \phi^a \sin \theta+ B_i^a  \phi^a \cos \theta) \dd[3]{x}
\end{equation}
%
and apply Stokes' theorem:
%
\begin{equation}
  E \geq \int_{\Sigma_\infty}
  \phi^a \qty(E_i^a  \sin \theta+ B_i^a   \cos \theta ) \dd{S_i}\,.
\end{equation}

Now, by the boundary conditions on \(\phi\) it will be radially outward \todo{Ti chiediamo, per nostra mancanza, che significa radially outward} and with absolute value \(a\) at infinity: therefore its product with the electric and magnetic fields (which are also radial at infinity) reduces to
%
\begin{equation}
  E \geq a (q\sin \theta + g \cos \theta)\,,
\end{equation}
%
where \(q\) and \(g\) are the electric and magnetic charges.
The sharpest bound occurs when the \((q, g)\) and \((\sin \theta, \cos \theta)\)  2D vectors are aligned, and corresponds to their euclidean scalar product: the final bound we get is thus found by the saturation of the Cauchy-Shwarz inequality: 
\begin{equation}
E \geq a \sqrt{q^2 + g^2} .
\end{equation}

In the 't Hooft-Polyakov monopole case, this simplifies to \(E \geq a \abs{g} \).

\subsubsection{Saturation of the Bogomol'nyi Bound}

A state which attains the lower bound is called a Bogomol'nyi – Prasad – Sommerfield --- for short BPS --- state.

To derive the bound we discarded always-positive quantities in the expression of the energy integrals; hence, in order to reach the bound they must be assumed to vanish everywhere. In addition, we assume our solution to be static: this implies that electric charge will be zero. This yields the conditions:
%
\begin{subequations}
\begin{align}
  V(\phi) &= 0 \label{eq:potential-condition}   \\
  D_0 \phi^a &= 0  \\
  E_i^a &= 0  \\
  B_i^a &= \pm D_i \phi^a  \label{eq:bogomolnyi-condition} \,.
\end{align}
\end{subequations}

Now, we make an interesting observation: if \(\lambda \neq 0\) in condition \eqref{eq:potential-condition} we must have that \(\phi^a \phi^a = a^2\) everywhere: therefore \(D_i (\phi^a \phi^a) = 2 \phi^a D_i \phi^a = 0\), and an analogous equation holds with \(D_i \rightarrow \partial_i\).
However, we also have equation \eqref{eq:bogomolnyi-condition}: using it we can also derive \(\phi^a D_i \phi^a \propto \phi^a B_i^a = 0\): the solution' magnetic field along \(\phi \) is trivial.
Therefore, in order to saturate the bound for a nontrivial solution we must assume \(\lambda = 0\).

If we wish to study this saturation despite this issue, we can insert the 't Hooft-Polyakov \emph{ansatz} in the Bogomol'nyi condition \eqref{eq:bogomolnyi-condition} we get the following set of differential equations:
%
\begin{subequations}
\begin{align}
  \xi \dv{K}{\xi}  &=  -KH   \\
  \xi \dv{H}{\xi} &= -K^2 + H + 1 \,;
\end{align}
\end{subequations}

these can be solved analytically, when coupled to the regular monopole equations of motion.

% \todo{Add computations to show that the energy density of the monopole is finite at the origin.}

\subsubsection{Numerical estimation}

We wish to give numerical values to our estimates of monopole masses.
We start from the 't Hooft-Polyakov monopole: from the Bogomol'nyi bound \(E \geq a \abs{g} \), while from the quantization condition we have that the least value \(g\) can attain is \(4 \pi / e\), where \(e\) is the elementary electric charge. 

In (Lorentz-Heaviside) natural units, we can express the vacuum expectation value of the Higgs field as \(a = M_W / e \), where \(M_W\) is the mass of the \(W\) boson. 

Thereforewe have \(E \geq 4 \pi a / e  = 4 \pi M_W / e^2 = M_W / \alpha \). 

Since \(M_W\) has been experimentally found to be around \SI{90}{GeV}, this means that \(E \gtrsim \SI{12}{TeV}\).

Some numerical simulations by Julia and Zee \cite{Julia:Dyon} for equations \eqref{eq:thooft-polyakov-equations}  have shown that numerical solutions to the monopole differential equations exist close to this bound, at \(E = 1.18 M_W / \alpha \).

They also simulated dyons according to equations \eqref{eq:dyon-equations}: they found \(E \approx  1.25 M_W/ \alpha \) for a dyon electric charge of \(Q \approx 44 e\) and \(E \approx 1.85 M_W / \alpha \) for \(Q \approx  169 e\).

This shows that, even while relaxing the condition of trivial potential one can get rather close to the BPS bound for the monopole mass.
For dyons we still have the BPS bound but the reasoning from before cannot be applied since we do not have the quantization condition which was derived only for monopoles.

\end{document}
